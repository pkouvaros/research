%%%% ijcai24.tex

\typeout{Towards Formal Verification of Neuro-symbolic Multi-agent Systems}

% These are the instructions for authors for IJCAI-23.

\documentclass{article}
\pdfpagewidth=8.5in
\pdfpageheight=11in

% The file ijcai23.sty is a copy from ijcai22.sty
% The file ijcai22.sty is NOT the same as previous years'
\usepackage{ijcai23}

\usepackage[T1]{fontenc}
\usepackage{amsmath,amssymb,mathtools}
\newcommand{\set}[1]{\left\{ #1 \right \}}


% Use the postscript times font!
\usepackage{times}
\usepackage{soul}
\usepackage{url}
\usepackage[hidelinks]{hyperref}
\usepackage[utf8]{inputenc}
\usepackage[small]{caption}
\usepackage{graphicx}
\usepackage{amsmath}
\usepackage{amsthm}
\usepackage{booktabs}
\usepackage{algorithm}
\usepackage{algorithmic}
\usepackage{xspace}
\usepackage[switch]{lineno}

\newcommand{\conexptime}{co\nexptime}
\newcommand{\nexptime}{\textsc{NExpTime}\xspace}
\newcommand{\venus}{\textsc{Venus}\xspace}
\newcommand{\mcmasp}{\textsc{MCMAS-P}\xspace}



% Comment out this line in the camera-ready submission
\linenumbers

\urlstyle{same}

% the following package is optional:
%\usepackage{latexsym}

% See https://www.overleaf.com/learn/latex/theorems_and_proofs
% for a nice explanation of how to define new theorems, but keep
% in mind that the amsthm package is already included in this
% template and that you must *not* alter the styling.
\newtheorem{example}{Example}
\newtheorem{theorem}{Theorem}

% Following comment is from ijcai97-submit.tex:
% The preparation of these files was supported by Schlumberger Palo Alto
% Research, AT\&T Bell Laboratories, and Morgan Kaufmann Publishers.
% Shirley Jowell, of Morgan Kaufmann Publishers, and Peter F.
% Patel-Schneider, of AT\&T Bell Laboratories collaborated on their
% preparation.

% These instructions can be modified and used in other conferences as long
% as credit to the authors and supporting agencies is retained, this notice
% is not changed, and further modification or reuse is not restricted.
% Neither Shirley Jowell nor Peter F. Patel-Schneider can be listed as
% contacts for providing assistance without their prior permission.

% To use for other conferences, change references to files and the
% conference appropriate and use other authors, contacts, publishers, and
% organizations.
% Also change the deadline and address for returning papers and the length and
% page charge instructions.
% Put where the files are available in the appropriate places.


% PDF Info Is REQUIRED.
% Please **do not** include Title and Author information
\pdfinfo{
/TemplateVersion (IJCAI.2023.0)
}

\title{Towards Formal Verification of Neuro-symbolic Multi-agent Systems}


% Single author syntax
\author{
    Panagiotis Kouvaros
    \affiliations
    Department of Computing, Imperial College London
    \emails
    p.kouvaros@imperial.ac.uk
}

% Multiple author syntax (remove the single-author syntax above and the \iffalse ... \fi here)
\iffalse
\author{
First Author$^1$
\and
Second Author$^2$\and
Third Author$^{2,3}$\And
Fourth Author$^4$
\affiliations
$^1$First Affiliation\\
$^2$Second Affiliation\\
$^3$Third Affiliation\\
$^4$Fourth Affiliation
\emails
\{first, second\}@example.com,
third@other.example.com,
fourth@example.com
}
\fi

\begin{document}

\maketitle

\begin{abstract}

This paper outlines some of the key methods we developed towards the formal
verification of multi-agent systems, covering both symbolic and connectionist
systems.  It discusses logic-based methods for the verification of unbounded
multi-agent systems (i.e., systems composed of an arbitrary number of
homogeneous agents, e.g., robot swarms), optimisation approaches for
establishing the robustness of neural network models, and methods for analysing
properties of neuro-symbolic multi-agent systems.

\end{abstract}

\section{Introduction}

Recent advances in Artificial Intelligence (AI) have enabled the automation of
challenging tasks, such as computer vision, that have been traditionally
difficult to tackle using classical approaches.  This accelerated the trend of
incorporating AI components in  diverse applications with high societal impact,
such as healthcare and transportation. 

Still, even though there is an increasing consensus in AI being beneficial for
society,  its inherent fragility and opacity hinders its adoption in
safety-critical applications. The associated risks are compounded in an
increasingly inter-connected, socio-techno-economic world, where systems of
multiple interacting intelligent agents, or multi-agent systems (MAS),
constitute a paradigm shift from object-oriented to interaction-oriented design
standards. 

In response to these concerns the area of formal verification of AI has grown
rapidly over the past few years to provide methods to automatically verify that
AI systems robustly behave as intended.

One of the key techniques that has emerged in the area is that of {\em model
checking}~\cite{Clarke+99a}. Model checking provides automated solutions to 
the problem of estabilishing whether a model $M_S$ representing a system $S$ satisfies a
logical formula $\phi_P$ encoding a specification $P$. In the case of MAS,
the formula $\varphi$ does not simply express temporal properties of systems, as
in reactive systems, but it may also be accounting for high-level attitudes of
agency, such as knowledge and strategies, which can be described in
temporal-epistemic logic~\cite{Fagin+95b} and alternating-time
logic~\cite{Alur+98a}.

Whilst a number of methods, such as binary binary decision
diagrams~\cite{GammieMeyden04a} and bounded model
checking~\cite{PenczekLomuscio03b},   enabled the  model checking  of complex systems of
large state spaces, a main drawback of the approach remains the {\em state-space
explosion problem}. Succinctly put, whereas model checking whether $M_S$
satisfies $\phi$ can be solved in polynomial time~\cite{Clarke+99a}, the size of the model $M_S$ to
analyse is exponential in the number of variables encoding the agents in the
system $S$.  

Notwithstanding that in practice this limits model checking to the verification
of systems with only few constituents, the analysis of  systems with arbitrarily many
participants, such as robot swarms and  applications in the Internet of Things, 
raises a principal barrier to the application of model checking.  Indeed, 
verifying systems of this kind, henceforth {\em unbounded multi-agent systems}
(UMAS), requires checking whether any system for any number of agents satisfies
the specification in question. This  renders model checking intractable when 
enumerating and analysing all individual systems. 


% The formal verification problem is concerned withV
% $S$ satisfies a safety property $P$. {\em Model checking}, a key method for the
% formal verification of reactive systems, has also been used in the past fifteen
% years to provide automated solutions to this problem~\cite{Clarke+99a}. In
% model checking, the system is represented as a model $M_S$, the specification
% is encoded as formula $\varphi$ and it is then checked whether $M_S$ satisfies
% $\varphi$. 

% In these cases, one could encode a system with a given number of agents and
% verify that a specification holds. However, additional agents may possibly
% interfere with the system in unpredictable ways resulting in the specifications
% being violated. Therefore, to fully verify the system, the process would have to
% be repeated for any possible number of components

Another key limitation of model checking is the requirement that the systems are
given in traditional and agent-based programming languages, thereby not accounting
for the recent shift to synthesise parts of the agents from data
and implement via neural networks. Systems of this kind, henceforth {\em
neuro-symbolic multi-agent systems} (NMAS), constitute important forthcoming
applications, such as autonomous vehicles and personal negotiation assistants,
where the neural components are employed to perform complex tasks, such as computer
vision and natural language processing.

While the efficacy of neural networks with respect to these tasks has
significantly improved over the past few years,  their fragility to adversarial
attacks~\cite{Szegedy+14} and lack of interpretability~\cite{VelizKim17} raise 
additional concerns regarding the overall system safety, thereby strengthening
the need for their principled analysis before deployment.

This paper gives an overview of the methods that we developed within the
Verification of Automous Systems research
group~\footnote{\url{https://vas.doc.ic.ac.uk}} towards the formal verification
of UMAS and NMAS.  Our pioneering work in the verification of UMAS, discussed in
Section~2, overcomes the model checking barrier with the development of methods
that enable the derivation of the number of agents that is sufficient to
consider when evaluating a specification. Our studies in the analysis of NMAS,
outlined in Section~3, include efficient methods for the verification of neural
networks and mixed integer programming formulations for checking system-level
specifications.  The paper concludes in Section~4 with directions for
future work.


\section{Unbounded Multi-agent Systems}

Interpreted systems are a standard semantics for describing multi-agent
systems~\cite{Fagin+95a}.  They provide a natural setup to interpret
specifications in a variety of languages including temporal-epistemic logic and
alternating temporal logic~\cite{Fagin+95b,LomuscioRaimondi06c}. Parameterised
Interpreted Systems (PIS)  is a parametric extension to interpreted systems
that we put forward  to reason about the temporal-epistemic properties of UMAS
in both synchronous~\cite{KouvarosLomuscio15b} and
asynchronous~\cite{KouvarosLomuscio16a} settings.   The parameter in PIS denotes
the number of agents in the system, each homogeneously constructed from an agent
template. 

The verification problem for PIS (generally known as the {\em parameterised
verification problem} in the reactive systems' literature~\cite{Bloem+15}) is to
check whether any system for any value of the parameter satisfies a given
specification.  This is in general undecidable~\cite{KouvarosLomuscio16a}.
Solutions to the problem can thus be given only in the form of incomplete techniques.
Alternatively, decidable fragments can be curved by imposing
restrictions on the systems and/or the specifications.  

In either case, a key concept that enables the verification of UMAS 
is that of a {\em cutoff}.  A cutoff is a natural number that expresses
the number of agents that is sufficient to consider when evaluating a given 
specification. In other words, if a cutoff can be computed, then the
verification problem can be solved by checking all systems whose number of
agents is below the cutoff value.

% ~\footnote{In addition to providing solutions to the
% verification problem, the identification of cutoffs can also be used to check
% whether the underlying system exhibits a certain {\em emergent behaviour} (i.e.,
% a behaviour that is realised  only when certain lower bounds on the number of
% agents are met) of interest.

% As we've shown cutoffs have a formal conncection to the emergent behaviours,
% i.e., behviours that are exchibited by the system only when certain lower bounds
% on the number of agents are satisfied. Intuitrevely, since any propertied (not)
% satisifed by the system with $c$ agents, where $c$ is a cutoff, is also (not)
% satisfied by any bigger system, a property is an emergent behviour if and only
% if it is satisfied by the cutoff system~\cite{KouvarosLomuscio15b}.

Whilst we've shown that cutoffs do not always exist~\cite{KouvarosLomuscio13b},
strong empirical evidence supports their existence for real-world
systems~\cite{EmersonKahlon00,EmersonNamjoshi95,Benjamin+14}. Moreover, for the
cases where they do not exist, theoretical analyses that we conducted show that
these often concern systems that can exhibit {\em impractical} cyclic behaviours
whose number of repetitions depends on the exact number of agents in the
system~\cite{KouvarosLomuscio13b}.

% verification queries for real-world systems an,
% specifications often admin a cutoff.  Theoretical analyses also show that
% systems that verification problems with no cutoffs concern peculiar behav do
% have a cutoff are systems of behaviours expressing a precise, arbitrarily large,
% number of agents in the system~\footnote{Roughly speaking, undecidability
% follows because of the expressiveness of the framework to represent the number
% of participants in the system, i.e., for any $n \in \mathbb{N}$, to construct a
% specification that is {\em true} only when the system comprised $n$ agent. For
% instance, if the execution of an action by an agent requires another agent to
% deadlock, then the number of times the action can be performed depends on the
% total number of agents in the system. This expressivity enables the simulation
% of the halting problem of a  2-counter machine.}. Such properties are not the
% typical properties to check, which are formed irrespective of the number of
% agents in the system (e.g, every agent in a system of any agents can never enter
% an unsafe configuration). 

Following these observations we have analysed various sufficient conditions for
the identification of cutoffs. The conditions were drawn with respect to
different  synchronisation primitives endowing the agents.  In the fully
synchronous setting, we have shown that cutoffs can always be identified and
gave a procedure for their computation~\cite{KouvarosLomuscio15b}.  In the
asynchronous case, where agents communicate via broadcast actions, we have
similarly given a sound and complete technique for their
derivation~\cite{KouvarosLomuscio13a}. When the agents
can additionally participate in pairwise communication with their environment,
we have shown that if 
\begin{itemize} \item[(i)] the environment can never block
pairwise synchronisations for the system of one agent only, and \item[(ii)] each
synchronisation can happen in unique configurations for the environment,
\end{itemize} then  cutoffs can be computed in an efficient procedure that runs
in linear time in the size of the agent template~\cite{KouvarosLomuscio13b}. The
second restriction can be lifted in a cutoff procedure that runs in exponential
time~\cite{KouvarosLomuscio15}. 

While these results were drawn with respect to {\em homogeneous} UMAS, where every
agent is instantiated from a unique agent template, we have also provided
extensions that account for {\em heterogeneous} UMAS, where agents can assume
different roles and responsibilities, e.g., heterogeneous robot
swarms~\cite{KouvarosLomuscio16a}. The heterogeneous semantics that we introduced allow for
broadcast actions that may either concern all agents of all agent templates or
all agents following a certain template. They additionally enable pairwise
interactions between agents of different roles, thereby surpassing the
expressive power of the homogeneous model.

Further gains in the expressivity of the protocols that can be verified have been
obtained from our studies on UMAS programmed using variables with infinite
domains~\cite{KouvarosLomuscio17a}. The resulting verification method  combines
predicate abstraction~\cite{LomuscioMichaliszyn15} with parameterised
verification, the former addressing the unboundedness of the state-space of the
agents and the latter tackling the unboundedness of their number.  



% Other
% extensions of PIS that we devised have been used to describe {\em open MAS},
% where countably many agents can join and leave the system at
% runtime~\cite{Kouvaros+19}.
% We have given verification
% methods for open MAS for both synchronous and asynchronous
% semantics~\cite{Kouvaros+19}.

We have released the open-source parameterised verification toolkit \mcmasp
implementing the cutoff procedures for UMAS.  \mcmasp enabled for the first time
the verification of aggregation and foraging algorithms for robot swarms
irrespective of the number of robots composing the
swarm~\cite{KouvarosLomuscio15b,KouvarosLomuscio16a}.  

Further applications included the analysis of the security of an unbounded
number of concurrent sessions of cryptographic protocols, for which we provided
a mapping from a Dolev-Yao threat model to
PIS~\cite{BoureanuKouvarosLomuscio16}.   
Others concerned  the verification of UMAS comprising {\em data-aware agents},
i.e., agents that are endowed with possibly infinite domains and that interact
with an environment composed of (semi)-structured
data~\cite{MontaliCalvaneseGiacomo14}.  Having used simulation-based
abstractions to deal with the infinity of the  agents, we have then presented  a
translation  to PIS to solve their verification
problem~\cite{BelardinelliKouvarosLomuscio17}. Analogous translations to PIS were 
 given for {\em open MAS},  where countably many agents can join and
leave the system at runtime~\cite{Kouvaros+19}.


Finally,  adaptations of the verification methods for PIS  enabled us to derive
techniques for the verification of opinion formation protocols in swarms, which
we used to give formal guarantees on the outcome of consensus
protocols~\cite{KouvarosLomuscio16b}. Still others facilitated the
verification of strategic properties of UMAS expressed in a parameterised
variant  of Alternating-time temporal logic that we
introduced~\cite{KouvarosLomuscio16c}.

I conclude this section by noting that complementary to protocol correctness,
which the aforementioned  methods can formally ascertain, the evaluation
of protocols also requires   analyses of the extent to which they are are
resilient to adverse functioning behaviours for some of the agents in the
system.  For instance, when evaluating a robot swarm search-and-rescue scenario,
it is not sufficient to establish that the swarm will collectively cover the
search area, but it is also crucial to determine that local faults, e..g,
hardware malfunctions, will be tolerated by the swarm, instead of being
propagated through agent interactions thereby dis-coordinating the search. To
address this concern we have put forward an automated procedure to establish the
robustness of UMAS against a given ratio of faulty to non-faulty agents in the
system~\cite{KouvarosLomuscio17b}, which we followed by a symbolic method to
automatically synthesise the maximum ratio of faulty to non-faulty agents a UMAS
can tolerate~\cite{KouvarosLomuscioPirovano18}.  



\section{Neuro-symbolic Multi-agent Systems}



To reason about the properties of NMAS, we have
introduced {\em neural interpreted systems} (NIS), a novel formalisation of MAS
based on interpreted systems. In a nutshell, an agent in NIS comprises a
perception mechanism implement via neural networks and coupled with a symbolic
action mechanism.  

The neural network components, which endow the agents with infinite domains,
pose significant  challenges to the verification problem.  In particular,
differently from traditional verification for symbolic systems, where atomic
formulae are evaluated in constant time at symbolic states of the system, the
evaluation of atomic formulae in NIS includes the computation of the output
regions of the neural networks  for a (potentially infinite) set of inputs,
which is an NP-complete problem~\cite{Katz+17}. Increasingly sophisticated
solutions to the problem have been put forward in the past few years  offering
a principled way of analysing the robustness of neural networks to adversarial
attacks~\cite{Singh+19,KouvarosLomuscio21,Wang+21}

We first discuss our work on the verification of standalone neural networks
which forms the backbone of our studies on the analysis of  NMAS outlined later.


\subsection{Neural network verification} The neural network verification problem is to
determine whether the output of a neural network is as expected for a
potentially infinite set of inputs. Perhaps the most common instantiation of the
problem is the adversarial robustness check for computer vision models, whereby
the set of inputs denotes imperceptible perturbations to a given image and the
set of outputs encodes classification equivalence for all perturbations. While
significant progress has been made in push-down engines to solve the
problem~\cite{Brix+23}, scalability to industrial-size models found in complex
tasks such as computer vision remains a key difficulty in the area.  

Advances in the area are driven by {\em complete} and {\em incomplete} methods.
Complete methods can in principle return a definite answer as to the whether the
verification problem is satisfied, whereas incomplete methods may be unable to
decide one way or the other. Complete methods are based on exact MILP/SMT
formulations SMT encodings or dedicated branch-and-bound procedures tackling the
optimisation mappings of the problem.  Incomplete methods typically rely on
linear/semi-definite relaxations of of the problem, which improves their
scalability over complete methods, but potentially hinders their efficacy to
solve the problem because of the induced over-approximations.

Our efforts in the area included methods towards improving scalability in
complete verification and techniques in the direction of strengthening precision
in incomplete verification. In complete verification, we have introduced the
novel concept of {\em ReLU dependency} whereby ReLU nodes are in a dependency
relation if their operational states for a set of inputs is connected by logical
implication.  We have devised methods for the computation of the dependency
relations, which we  have translated into MILP cuts, thereby improving the
efficacy of MILP formulations of the verification problem~\cite{Botoeva+20}.
Further improvements have been possible via the derivation of dependency-based
branching heuristics  in a branch-and-bound procedure~\cite{KouvarosLomuscio21}.


In incomplete verification, we have devised an abstraction method that
strengthens the linear relaxations of ReLU nodes by additionally accounting for
intra-layer dependencies, instead of simply the local over-approximation areas,
when choosing a relaxation~\cite{HashemiKouvarosLomuscio21}. This enabled proofs
of adversarial robustness to be given to computer vision models that could not
be previously resolved. To a similar effect we have strengthen semidefinite
approximations of the verification problem  considering layer-wise relaxations
and incorporating linear cuts into the semidefinite formulation.

We have released the open-source neural network verification toolkit \venus
implementing the aforementioned methods.  In the span of four years, \venus has
progressed from analysing networks of few thousands of nodes to examining
networks of millions of nodes. Among the latter are neural network-based systems
in the aircraft domain developed by Boeing, including object detection and
landing assistance systems~\cite{Kouvaros+21}, and semantic segmentation models
for pose estimation~\cite{Kouvaros+23}.


\subsection{NMAS verification} The scalability hurdle is even more prevalent in the
formal analysis of closed-loop systems with neural components such as NMAS. In
fact, we've shown their verification problem to be undecidable for plain
reachability properties~\cite{Akintunde+22}. In the light of this we've isolated
bounded fragments of CTL and ATL,  whereby formulae can be evaluated in a
bounded number of execution steps, thus allowing for expressing properties
concerning, for instance, whether the agents can bring about a state of affairs
or reach a safe configuration within a bounded number of steps. We have shown
that verification with respect to this fragment is in \conexptime and solved the
resulting verification problem via a mixed-integer linear programming (MILP)
formulations.  that encapsulate the neural network verification procedures
described earlier for tight encodings of the
networks~\cite{Akintunde+20,Akintunde+20b}. 
To further alleviate the difficulty of the verification problem, we
also introduce a novel algorithm that checks for the occurrence of
bugs in parallel over the execution paths. As we show, in the case
of bounded safety specifications, this enables us to return a bug to
the user as soon as a violation is identified on any of the branching
paths that are explored in parallel.

% monolithic and compositional verification algorithms


\section{Conclusions and Future Work}



\section*{Acknowledgments}

I am grateful to the Verification of Autonomous Systems research group, Imperial
College London, for the mentoring and support in conducting the research
presented in this paper. I acknowledge the financial support of the DARPA
Assured Autonomy Programme (FA8750-18-C-0095) and the EPSRC Research Project
Trusted Autonomous Systems (EP/I00520X/1). Finally, I thank the IJCAI 2023 program
committee for the  invitation to deliver a spotlight talk in the Early Career
Spotlight Track.



%% The file named.bst is a bibliography style file for BibTeX 0.99c
\bibliographystyle{named}
\bibliography{pk}

\end{document}

