%%%% ijcai24.tex

\typeout{Towards Formal Verification of Neuro-symbolic Multi-agent Systems}

% These are the instructions for authors for IJCAI-23.

\documentclass{article}
\pdfpagewidth=8.5in
\pdfpageheight=11in

% The file ijcai23.sty is a copy from ijcai22.sty
% The file ijcai22.sty is NOT the same as previous years'
\usepackage{ijcai23}

\usepackage[T1]{fontenc}
\usepackage{amsmath,amssymb,mathtools}
\newcommand{\set}[1]{\left\{ #1 \right \}}


% Use the postscript times font!
\usepackage{times}
\usepackage{soul}
\usepackage{url}
\usepackage[hidelinks]{hyperref}
\usepackage[utf8]{inputenc}
\usepackage[small]{caption}
\usepackage{graphicx}
\usepackage{amsmath}
\usepackage{amsthm}
\usepackage{booktabs}
\usepackage{algorithm}
\usepackage{algorithmic}
\usepackage[switch]{lineno}

% Comment out this line in the camera-ready submission
\linenumbers

\urlstyle{same}

% the following package is optional:
%\usepackage{latexsym}

% See https://www.overleaf.com/learn/latex/theorems_and_proofs
% for a nice explanation of how to define new theorems, but keep
% in mind that the amsthm package is already included in this
% template and that you must *not* alter the styling.
\newtheorem{example}{Example}
\newtheorem{theorem}{Theorem}

% Following comment is from ijcai97-submit.tex:
% The preparation of these files was supported by Schlumberger Palo Alto
% Research, AT\&T Bell Laboratories, and Morgan Kaufmann Publishers.
% Shirley Jowell, of Morgan Kaufmann Publishers, and Peter F.
% Patel-Schneider, of AT\&T Bell Laboratories collaborated on their
% preparation.

% These instructions can be modified and used in other conferences as long
% as credit to the authors and supporting agencies is retained, this notice
% is not changed, and further modification or reuse is not restricted.
% Neither Shirley Jowell nor Peter F. Patel-Schneider can be listed as
% contacts for providing assistance without their prior permission.

% To use for other conferences, change references to files and the
% conference appropriate and use other authors, contacts, publishers, and
% organizations.
% Also change the deadline and address for returning papers and the length and
% page charge instructions.
% Put where the files are available in the appropriate places.


% PDF Info Is REQUIRED.
% Please **do not** include Title and Author information
\pdfinfo{
/TemplateVersion (IJCAI.2023.0)
}

\title{Towards Formal Verification of Neuro-symbolic Multi-agent Systems}


% Single author syntax
\author{
    Panagiotis Kouvaros
    \affiliations
    Department of Computing, Imperial College London
    \emails
    p.kouvaros@imperial.ac.uk
}

% Multiple author syntax (remove the single-author syntax above and the \iffalse ... \fi here)
\iffalse
\author{
First Author$^1$
\and
Second Author$^2$\and
Third Author$^{2,3}$\And
Fourth Author$^4$
\affiliations
$^1$First Affiliation\\
$^2$Second Affiliation\\
$^3$Third Affiliation\\
$^4$Fourth Affiliation
\emails
\{first, second\}@example.com,
third@other.example.com,
fourth@example.com
}
\fi

\begin{document}

\maketitle

\begin{abstract}

This paper outlines some of the key methods we developed towards the formal
verification of multi-agent systems, covering both symbolic and connectionist
systems.  It discusses logic-based methods for the verification of unbounded
multi-agent systems (i.e., systems composed of an arbitrary number of
homogeneous agents, e.g., robot swarms), optimisation approaches for
establishing the robustness of neural network models, and methods for analysing
properties of neuro-symbolic multi-agent systems.

\end{abstract}

\section{Introduction}

Significant advances in Artificial Intelligence (AI) have enabled the
automation of challenging tasks, such as computer vision, that have been
traditionally difficult to tackle using classical approaches.  This accelerated
the trend of incorporating AI components in diverse applications with high
societal impact, such as healthcare and transportation. Still, even though
there is an increasing consensus in AI being beneficial for society,  its
inherent fragility hinders its adoption in safety-critical applications.  In
response to these concerns the area of formal verification of AI has grown
rapidly over the past few years to provide methods to automatically verify that
AI systems robustly behave as intended.

The formal verification problem is concerned with establishing whether a MAS
$S$ satisfies a safety property $P$. {\em Model checking}, a key method for the
formal verification of reactive systems, has also been used in the past fifteen
years to provide automated solutions to this problem~\cite{Clarke+99a}. In
model checking, the system is represented as a model $M_S$, the specification
is encoded as formula $\varphi$ and it is then checked whether $M_S$ satisfies
$\varphi$. In the case of MAS, the formula $\varphi$, does not simply express
temporal properties of systems, as in reactive systems, but it also denotes
high-level attitutes of agency, such as knowledge and strategies, as these are
expressed in temporal-epistemic logic~\cite{Fagin+95b} and alternating-time
logic~\cite{Alur+98a}.

A number of methods were put forward in the area that enabled the computation of
the model checking query for progressively bigger systems. These include binary
decision diagrams [Gammie and Meyden, 2004; Lomuscio et al., 2009], abstraction
[Co- hen et al., 2009], partial order reduction [Lomuscio et al., 2010] and
bounded model checking [Lomuscio et al., 2007].

Even though the methods have enabled to model checking of complex systems of
very large state spaces, model checking limits the formal verification of MAS to
(i) systems with a known number of participants at design time; (ii) systems
with purely symbolic components. This is in contrast to the current trend of
developing and deploying MAS with an unbounded number of participants, as in
robot swarms, multi-party negotiation protocols and auctions, voting protocols
and e-sevices. 

Unbounded Multi-agent Systems (UMAS) are MAS composed of homogeneous agents,
each instantiated by a unique {\em agent template}, whose number in not known at
design time. 

In contrast to traditional models of agency, where the agent’s behaviour is
given in an agent-based programming language, these methods do not accounts for
the recent shift to synthesise the agents’ behaviour from data.



the methods cannot in principle completely overcome the {\em state-space
explosion problem}, a key limitation of model checking whereby the state-space
is exponential in the number of variables encoding the system to be checked. 

State space explosion problem

Purely symbolic agents. 

In these cases, one could encode a system with a given number of agents and
verify that a specification holds. However, additional agents may possibly
interfere with the system in unpredictable ways resulting in the specifications
being violated. Therefore, to fully verify the system, the process would have to
be repeated for any possible number of components

\section{Unbounded Multi-agent Systems}

Interpreted systems is a main semantical structure for the formal description of
multi-agent systems and the interpretation of agent-based specifications,
including those expressed in temporal-epistemic logic and alternate time
logic~\cite{Fagin+95b,LomuscioRaimondi06c}. Parameterised Interpreted Systems
(PIS)  is an extension to interpreted systems  that we put forward  to reason
about the temporal-epistemic properties of UMAS in both
synchronous~\cite{KouvarosLomuscio15b} and
asynchronous~\cite{KouvarosLomuscio16a} settings.   The parameter in PIS denotes
the number of agents in the system, each homogeneously constructed from an agent
template. 

The verification problem for PIIS (generally known as the {\em parameterised
verification problem} in the reactive systems' literature~\cite{Bloem+15} is to
check whether any system, for any value of the parameter, satisfies a given
specification.  This is in general undecidable~\cite{KouvarosLomuscio16a}.
General solutions can thus be given only in the form of incomplete techniques.
Alternatively, decidable fragments of the problem can be curved by imposing
restrictions on the systems and/or the specifications.  

A key notion that enables the construction of verification methods in both
settings is that of a {\em cutoff}.  A cutoff is a natural number that expresses
the number of components that is sufficient to analyse when evaluating a given 
specification. In other words, if a cutoff can be computed, then the
verification problem can be solved by checking all systems whose number of
agents is below the cutoff value.  In addition to providing solutions to the
verification problem, the identification of cutoffs can also be used to check
whether the underlying system exhibits a certain {\em emergent behaviour} (i.e.,
a behaviour that is realised  only when certain lower bounds on the number of
agents are met) of interest.

% As we've shown cutoffs have a formal conncection to the emergent behaviours,
% i.e., behviours that are exchibited by the system only when certain lower bounds
% on the number of agents are satisfied. Intuitrevely, since any propertied (not)
% satisifed by the system with $c$ agents, where $c$ is a cutoff, is also (not)
% satisfied by any bigger system, a property is an emergent behviour if and only
% if it is satisfied by the cutoff system~\cite{KouvarosLomuscio15b}.

Although in theory  cutoffs do not always exist~\cite{KouvarosLomuscio13b},
strong empirical evidence supports their existence for real-world
systems~\cite{EmersonKahlon00,EmersonNamjoshi95,Benjamin+14}.  For the cases
where they do not exist, theoretical analyses show that these often concern
impractical cyclic behaviours whose number of repetitions depends on the exact
number of agents in the system~\cite{KouvarosLomuscio13b}.

% verification queries for real-world systems an,
% specifications often admin a cutoff.  Theoretical analyses also show that
% systems that verification problems with no cutoffs concern peculiar behav do
% have a cutoff are systems of behaviours expressing a precise, arbitrarily large,
% number of agents in the system~\footnote{Roughly speaking, undecidability
% follows because of the expressiveness of the framework to represent the number
% of participants in the system, i.e., for any $n \in \mathbb{N}$, to construct a
% specification that is {\em true} only when the system comprised $n$ agent. For
% instance, if the execution of an action by an agent requires another agent to
% deadlock, then the number of times the action can be performed depends on the
% total number of agents in the system. This expressivity enables the simulation
% of the halting problem of a  2-counter machine.}. Such properties are not the
% typical properties to check, which are formed irrespective of the number of
% agents in the system (e.g, every agent in a system of any agents can never enter
% an unsafe configuration). 

We have analysed various sufficient conditions for the identification of cutoffs
with respect to different  synchronisation primitives endowing the agents.  In
the fully synchronous setting, we have shown that cutoffs can always be
identified and gave a procedure for their
computation~\cite{KouvarosLomuscio15b}.
In the asynchronous case, where agents communicate via broadcast actions, we
have similarly given a sound and complete technique for their
identification~\cite{KouvarosLomuscio13a}. For the instances where the agents
can additionally participate in pairwise communication with their environment,
we have shown that if 
\begin{itemize}
	\item[(i)] the environment can never block pairwise synchronisations
for the system of one agent only, and
	\item[(ii)] each synchronisation can happen in unique
configurations for the environment,
\end{itemize}
then  cutoffs can be computed in an efficient procedure that runs in linear time
in the size of the agent template~\cite{KouvarosLomuscio13b}. The second
restriction can be lifted in a cutoff procedure that runs in exponential
time~\cite{KouvarosLomuscio15}. 

While the results were drawn with respect to {\em homogeneous} UMAS, where every
agent is instantiated from a unique agent template, we have also provided
extensions that account for {\em heterogeneous} UMAS, where agents can assume
different roles and responsibilities, e.g., heterogeneous robot
swarms~\cite{KouvarosLomuscio16a}. The heterogeneous semantics allow for
broadcast actions that may either concern all agents of all agent templates or
all agents following a certain template. Additionally, they enable pairwise
interactions between agents of different roles, thereby surpassing the
expressive power of the homogeneous model.

Further gains in the expressivity of protocols that can be verified have been
obtained by the verification method we introduced for UMAS programmed using
variables with infinite domains~\cite{KouvarosLomuscio17a}. The method  combines
predicate abstraction~\cite{LomuscioMichaliszyn15} with parameterised
verification (the former addressing the unboundedness of the state-space of the
agents and the latter tackling the unboundedness of their number).  Other
extensions of PIS have been used to describe {\em open MAS}, where countably
many agents can join and leave the system at runtime. We have given verification
methods for open MAS for both synchronous and asynchronous
semantics~\cite{Kouvaros+19}.

We have released the open-source parameterised verification toolkit
\textsc{MCMAS-P} implementing the aforementioned cutoff procedures.
\textsc{MCMAS-P} enabled for the first time the verification of aggregation and
foraging algorithms for robot swarms irrespective of the number of robots
composing the swarm~\cite{KouvarosLomuscio15b,KouvarosLomuscio16a}.  Further
applications included the analysis of the security of an unbounded number of
concurrent sessions of cryptographic protocols, for which we provided a mapping
from a Dolev-Yao threat model to PIS~\cite{BoureanuKouvarosLomuscio16}.  Others
concerned  the verification of UMAS with {\em data-aware agents}, i.e., agents
that are endowed with possibly infinite domains and that interact with an
environment composed of (semi)-structured data~\cite{MontaliCalvaneseGiacomo14}.
For this class of UMAS we similarly  gave a mapping to
PIS~\cite{BelardinelliKouvarosLomuscio17}.  Finally,  adaptations of the
counter-abstraction methods for PIS  enabled us to derive methods for the
verification of opinion formation protocols in swarms, which we used to give
formal guarantees on the outcome of consensus protocols. Other adaptations
facilitated the verification of strategic properties of UMAS expressed in a
parameterised variant  of Alternating-time temporal logic~\cite{Alur+98a} that
we introduced~\cite{KouvarosLomuscio16c}.

We conclude this section by noting that complementary to protocol correctness,
which the aforementioned cutoffs methods can formally ascertain, the evaluation
of protocols also requires   analyses of the extent to which they are are
resilient to adverse functioning behaviours for some of the agents in the
system.  For instance, when evaluating a robot swarm search-and-rescue scenario,
it is not sufficient to establish that the swarm will collectively cover the
search area, but it is also crucial to determine that local faults, e..g,
hardware malfunctions, will be tolerated by the swarm, instead of being
propagated through agent interactions thereby dis-coordinating the search. To
address this concern we have put forward an automated procedure to establish the
robustness of UMAS against a given ratio of faulty to non-faulty agents in the
system~\cite{KouvarosLomuscio17b}, which we followed by a symbolic method to
automatically synthesise the maximum ratio of faulty to non-faulty agents
before the robustness of UMAS is violated~\cite{KouvarosLomuscioPirovano18}.  



\section{Neuro-symbolic Multi-agent Systems}



To account for the recent shift to synthesise agents from data, instead of
agent-based programming languages, we have introduced a novel formalisation of
MAS, which we  call {\em Neuro-symbolic MAS} (NMAS). In a nutshell, an agent in
NMAS, comprises a perception mechanism implement via ReLU-based neural networks,
coupled with a symbolic action mechanism.  

The neural network components, which endow the agents with infinite domains,
pose significant  challenges to the verification of NMAS.  In particular, in
contrast to traditional verification for symbolic multiagent systems, where
atomic formulae are evaluated in constant time at symbolic states of the system,
the evaluation of atomic formulae in NMAS concerns the computation of the output
regions of the neural networks  for a (potentially infinite) set of inputs.
This atomic check is an NP-complete problem~\cite{Katz+17}. It has received
considerable attention in the past five years in the context of standalone
neural-network models as a principled way of analysing the inherent fragility of
neural networks to adversarial attacks.  

We first discuss our work on verifying standalone neural networks and then our
studies on analysing NMAS.


{\bf Neural network verification.} Definition. 
Perhaps the most significant instantiation of the problem is adversarial rob.
It is an NP-complete problem~\cite{Katz+17}. 
While progress has been made, a key difficulty in the area is scalability. Simply put,
the present methods, while effective on small models, can-
not presently analyse the networks used in vision and other
complex tasks. The approaches that have driven this effort are complete and incomplete.
ormal verification of neural networks comprises complete
and incomplete methods. Complete methods can in principle
return a definite answer as to whether the verification prop-
erty is satisfied, whereas incomplete methods may be un-
able to decide whether the property is satisfied. Complete
methods are based on MILP formulations [Botoeva et al.,
2020; Bastani et al., 2016; Lomuscio and Maganti, 2017;
Cheng et al., 2017; Fischetti and Jo, 2018; Tjeng et al.,
2019], SMT encodings [Ehlers, 2017; Katz et al., 2017;
Katz et al., 2019], and input refinement [Wang et al., 2018;
P. Henriksen, 2020]. Incomplete methods are based on du-
ality [Dvijotham et al., 2018; Wong and Kolter, 2018], lin-
ear approximations [Tran et al., 2020; Singh et al., 2019;
Weng et al., 2018; Tjandraatmadja et al., 2020] and semi-
definite relaxations [Fazlyab et al., 2020; Dathathri et al.,
2020]. While incomplete approaches differ, they all rely on
approximations of the ReLU function. This often improves
their scalability over complete methods but can also hinder
their efficacy to solve the verification problem.

Improving scalability in complete verification and precision in incomplete
verification. Introduced the novel notion of a dependency between nodes which we
used to refuced the configurations of the operations of the nodes in the network
that need to be analyses when solving the everification problem. We did this for
as cuts and as branching heuristic in branch-and-bound procedure.  rval
propagation in which the choice of the ReLU relaxation at each node is
determined via optimisation

We have released the state-of-the-art, open-source toolkit {\textsc Venus}
implementing these methods.  {\textsc Venus} has been used to analyse  Neural
Network-Based Systems in the Aircraft Domain developed by Boeing.  an object
detection system trained for open category detection and (ii.) a neural
controller trained to assist landing in non-towered airports. Venus is used to
identify the conditions under which these systems satisfy local robustness or
generate counterexamples that comprehensively show the circumstances under which
safety cannot be guaranteed

{\bf NMAS verification.}
Scalability - more evident when full systems are considered.
The overall verification
problem against CTL properties is undecidable~\cite{Akintunde+22}. Decidable
fragments can be obtained by the consideration of bounded properties, i.e.,

We develop and solve the
resulting verification problem via a mixed-integer linear pro-
gramming (MILP) formulation, present a tool developed for
this task and evaluate the approach on an avionics advisory
system for collision avoidance.

formulae of bounded CTL build
upon temporal modalities indexed with natural numbers denoting
the temporal depth up to which the formula is evaluated

MILP translations. Parallel executions 

o further alleviate the difficulty of the verification problem, we
also introduce a novel algorithm that checks for the occurrence of
bugs in parallel over the execution paths. As we show, in the case
of bounded safety specifications, this enables us to return a bug to
the user as soon as a violation is identified on any of the branching
paths that are explored in parallel.

monolithic and compositional verification algorithms

to coNE xp T imE upper bound for bounded CTL.

As neural networks are functions
over the reals, the states of the agents have infinite domains, as opposed to
finite ones in standard MAS. 
Starting from an initial state the
sequence of joint actions of the agents in a NIS induces a computation tree where 
the evaluation of what holds true at  each node of the tree, is not a simple
check in constant time set inclusion, but it involves the non-trivial check of
whether the output of neural networks satisfies some linear constraints for an
infinite set of inputs.  --- Accounting for real-valued inputs. This is  an
NP-complete problem. 





\section{Conclusions}



\section*{Acknowledgments}
VAS group.

%% The file named.bst is a bibliography style file for BibTeX 0.99c
\bibliographystyle{named}
\bibliography{pk}

\end{document}

